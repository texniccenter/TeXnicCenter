%%%%%%%%%%%%%%%%%%%%%%%%%%%%%%%%%%%%%%%%%%%%%%%%%%%%%%%%%%%%%%%%%%%%%%%%
%% $Id$
%%%%%%%%%%%%%%%%%%%%%%%%%%%%%%%%%%%%%%%%%%%%%%%%%%%%%%%%%%%%%%%%%%%%%%%%
% This file and the files in this directory and its subdirectories
% are intended to test several parts of the TeXnicCenter-system.
%
% Copyright (C) 2002-$CurrentYear$ ToolsCenter

% Their main purpose is to reproduce several bugs or behaviours coming
% up from missing features. They are neither a good starting point for
% working with TeX nor with the TeXnicCenter-system. If you use them,
% you do this on your own risk. They come WITHOUT ANY WARRANTY;
% without even the implied warranty of MERCHANTABILITY or
% FITNESS FOR A PARTICULAR PURPOSE.
%
% Anything below the "end of prolog"-line is for testing purposes only
% and does not reflect the opinions of the author(s) and is not meant
% to be a statement at all; it is even not said to be true or reliable.
%
% If you have further questions or if you want to support
% further TeXnicCenter development, visit the TeXnicCenter-homepage
%
%     http://www.ToolsCenter.org
%
% end of prolog %%%%%%%%%%%%%%%%%%%%%%%%%%%%%%%%%%%%%%%%%%%%%%%%%%%%%%%%


%%%%%%%%%%%%%%%%%%%%%%%%%%%%%%%%%%%%%%%%%%%%%%%%%%%%%%%%%%%%%
%% HEADER
%%%%%%%%%%%%%%%%%%%%%%%%%%%%%%%%%%%%%%%%%%%%%%%%%%%%%%%%%%%%%
\documentclass[a4paper,twoside,10pt]{report}

\begin{document}


A good place to start testing regular expressions
is the website \verb|https://regex101.com|.
Use ECMAScript-syntax for testing,
since the structure parser uses this version.


%%A user command for an input of a tex-file
% TXCUserCommand: InputFileTeXParse
% Name: myinput
% Description: augmenting the input command with some extras
% \\myinput\s*\{\s*\"?([^\}]*)\"?\s*\}
% Filename: 1
% 
\NewDocumentCommand{\myinput}{m}{\input{#1}}



%This should then be detected by the structure parser
\myinput{section}%comment after command

%This should NOT be detected by the structure parser
\MyInput{nosection}%comment after command


%%A user command for a heading
% TXCUserCommand: Heading           			   												
% Name: mysection
% Description: augmenting the section command with some extras
% \\mysection\*?\s*\{\s*(.*)\s*\}
% Title: 1
% Level: section
%
\NewDocumentCommand{\mysection}{m}{\section{#1}}

%This should then be detected by the structure parser
\mysection{A visible section}%comment after command

%This should NOT be detected by the structure parser
\MySection{A LaTeX error and the structure parser should not see it}%comment after command






\end{document}

